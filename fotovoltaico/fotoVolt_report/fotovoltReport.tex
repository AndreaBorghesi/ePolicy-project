% survey on photovoltaic installed power italian and regional trends 


\documentclass[12pt,a4paper,openright,twoside]{report}
\usepackage{graphicx}
\usepackage{todonotes}
\usepackage{eurosym}


\graphicspath{{./graphs/}}

\newcommand{\AB}[1]{\todo[inline,color=cyan]{{\bf AB:} #1}}

\begin{document}

\title{A SURVEY OF NATIONAL AND REGIONAL TRENDS IN PHOTOVOLTAIC POWER GENERATION IN ITALY}
\author{}
\date{}

\maketitle

\AB{A brief introduction - what we're going to describe in this report}

\section*{National Production}

The production of energy from photovoltaic, also referred to as PV, has increased in Italy since the introduction of  national incentives in 2006 (``\emph{Primo Conto Energia}''). The number of PV new panels installed each year, and  their cumulative power, has also grown until 2011 – when there was a peak in the installed power – while in 2012 we observe a relative drop in the kW of power coming from new installed plants. We can see the installed power trend on national level in Figure \ref{italiaNatTrend}.
The pattern we notice in Figure \ref{italiaNatTrend} is probably due to the changes in the incentives offered by the Italian government to those willing to invest in photovoltaic plants and the regulation concerning the plants – for example the permission to install the panels directly onto the ground, better called \emph{free-standing PV arrays}.

\begin{figure}[h]
	\centering
	\includegraphics[width=0.9\textwidth]{italiaNatTrend}
	\caption{Italy PV installed power}
	\label{italiaNatTrend}
\end{figure}

The incentive mechanism adopted to foster the PV energy generation in Italy is called ``Conto Energia'' and is based on feed-in tariffs (a fixed and guaranteed price paid to the eligible producers of electricity from renewable sources for the power they feed into the grid); different tariffs are offered to plants differing in size, technology employed, etc. 
This mechanism was introduced for the first time in 2005 with the so called ``Primo Conto Energia'' (28/07/2005 and 06/02/2006 Decrees), which launched the production-based tariffs, in imitation of the similar effective subsidies already adopted in Germany, and replaced the previous national investment grant given to cover the initial cost of the plant. Although the tariffs depend on various parameters regarding the plant, the values offered ranged from 0.460\euro/kWh to 0.490\euro/kWh.

After that, in 2007 started the so called ``Secondo Conto Energia'' (Ministerial Decree 19/02/2007) which lasted until 2010 - its termination was later postponed to 2011 - and introduced new tariffs ranges (between 0.360\euro/kWh and 0.490\euro/kWh) along with some new criteria devised to foster the production of electricity from PV sources. The following mechanism ``Terzo Conto Energia'' (D.M. 06/08/2010) was applied to the plants which started their activity between 01/01/2011 and 31/05/2011 and guaranteed tariffs ranging from 0.172\euro/kWh to 0.402\euro/kWh.
Shortly afterwards the so called ``Quarto Conto Energia'' became effective (precisely from 01/06/2011 onwards, D.M. 05/05/2011) with tariffs between 0.133\euro/kWh and 0.274\euro/kWh. The last mechanism implemented so far, ``Quinto Conto Energia'' (D.M. 05/07/2012), concerns plant starting their activity in 2013 and further lowers the tariffs, 0.010\euro/kWh and 0.208\euro/kWh. 

In order to provide the incentives to those willing to install PV panels, the Italian government has invested a large amount of financial resources, with a magnitude of approximately few billions of euros each year – more precisely the yearly cumulative cost of all the PV subsidies is expected to reach the value of 6.7\euro\ billions more or less around July 2013 and after this date the national incentives in Italy will cease. In any case the presence of  the incentive mechanisms provided in the past few years (since the introduction of feed-in tariffs) has effectively supported the adoption of the relatively new solar photovoltaic energy production, granting much better results than the previous mechanism, as we can easily see from the national trend shown in Figure \ref{italiaNatTrend}.

\begin{figure}[h]
	\centering
	\includegraphics[width=0.9\textwidth]{italiaClassiPot}
	\caption{Italy PV installed power, grouped by plant class}
	\label{italiaClassiPot}
\end{figure}


In order to give a better insight about the tendency shown in the previous graph, it could be useful to consider the differences in installed PV power for each plant class. In Italy are currently recognized six plant classes: \emph{class 1}, ``classe 1'', which comprises plant with a power ranging from 1kW to 3kW; \emph{class 2}, from 3kW to 20kW; \emph{class 3}, from 20kW to 200kW; \emph{class 4}, from 200kW to 1000kW; \emph{class 5}, from 1000kW to 5000kW; \emph{class 6}, with power greater than 5000kW. The installed power divided by plant class is shown in Figure \ref{italiaClassiPot}; as we see, there is quite a resemblance with the Figure \ref{italiaNatTrend}, with a rapid increase until 2011 (almost exponential) and a clear slowdown in 2012, and we can also note that there is a steeper surge, and consequent decrease, for the classes with larger power ranges, that is 4,5 and 6\footnote{A partial explanation could be that in 2011 the Italian government ceased to give incentives to panels directly put on the ground, which are usually larger than rooftop-mounted ones, fearing that they were subtracting areas to cultivated lands due to the excessively high feed-in tariffs}.


\section*{Regional Production}

In the remaining pages of this report we are going to show the trend of the installed energy production for each Italian region. For every region we exhibit 

Firstly, we noticed that almost every region in the past years has implemented some kind of incentive to foster the installation of PV panels, on top of those already provided by the Italian government. The large majority of regional subsidies were given in the form of investment grant, i.e. the Region pays a fraction of the total cost of the plan installation to the private investors, distributing the allocated funds until exhaustion. Usually the amount of financial resources made available was approximately a few millions of euros, often less than 5\euro\ millions and very rarely reaching values greater than 10\euro\ millions of euros. 

The large gap between the magnitude of national and regional subsidies (the regional ones are far smaller, even if we consider them together) suggested us that the former may have a much greater influence on the adoption of PV technologies on a national scale. The preliminary results we are going to present now give us a hint that this may well be the case, or at least lead us to think that the regional subsidies affected the PV installed power only in a marginal way.

\subsection*{Emilia-Romagna}

\begin{figure}[hp]
	\centering
	\includegraphics[width=0.9\textwidth]{emiliaromagnaRegTrend}
	\caption{Emilia-Romagna PV installed power}
	\label{emiliaromagnaRegTrend}
\end{figure}

\begin{figure}[hp]
	\centering
	\includegraphics[width=0.9\textwidth]{emiliaromagnaRegNatTrend}
	\caption{Emilia-Romagna and Italy trends}
	\label{emiliaromagnaRegNatTrend}
\end{figure}

\begin{figure}[hp]
	\centering
	\includegraphics[width=0.9\textwidth]{emiliaromagnaClassiPot}
	\caption{Emilia-Romagna PV installed power, grouped by plant class}
	\label{emiliaromagnaClassiPot}
\end{figure}


\subsection*{Lombardia}

\begin{figure}[hp]
	\centering
	\includegraphics[width=0.9\textwidth]{lombardiaRegTrend}
	\caption{Lombardia PV installed power}
	\label{lombardiaRegTrend}
\end{figure}

\begin{figure}[hp]
	\centering
	\includegraphics[width=0.9\textwidth]{lombardiaRegNatTrend}
	\caption{Lombardia and Italy trends}
	\label{lombardiaRegNatTrend}
\end{figure}

\begin{figure}[hp]
	\centering
	\includegraphics[width=0.9\textwidth]{lombardiaClassiPot}
	\caption{Lombardia PV installed power, grouped by plant class}
	\label{lombardiaClassiPot}
\end{figure}

\subsection*{Veneto}

\begin{figure}[hp]
	\centering
	\includegraphics[width=0.9\textwidth]{venetoRegTrend}
	\caption{Veneto PV installed power}
	\label{venetoRegTrend}
\end{figure}

\begin{figure}[hp]
	\centering
	\includegraphics[width=0.9\textwidth]{venetoRegNatTrend}
	\caption{Veneto and Italy trends}
	\label{venetoRegNatTrend}
\end{figure}

\begin{figure}[hp]
	\centering
	\includegraphics[width=0.9\textwidth]{venetoClassiPot}
	\caption{Veneto PV installed power, grouped by plant class}
	\label{venetoClassiPot}
\end{figure}

\subsection*{Umbria}

\begin{figure}[hp]
	\centering
	\includegraphics[width=0.9\textwidth]{umbriaRegTrend}
	\caption{Umbria PV installed power}
	\label{umbriaRegTrend}
\end{figure}

\begin{figure}[hp]
	\centering
	\includegraphics[width=0.9\textwidth]{umbriaRegNatTrend}
	\caption{Umbria and Italy trends}
	\label{umbriaRegNatTrend}
\end{figure}

\begin{figure}[hp]
	\centering
	\includegraphics[width=0.9\textwidth]{umbriaClassiPot}
	\caption{Umbria PV installed power, grouped by plant class}
	\label{umbriaClassiPot}
\end{figure}

\subsection*{Trentino Alto Adige}

\begin{figure}[hp]
	\centering
	\includegraphics[width=0.9\textwidth]{trentinoRegTrend}
	\caption{Trentino Alto Adige PV installed power}
	\label{trentinoRegTrend}
\end{figure}

\begin{figure}[hp]
	\centering
	\includegraphics[width=0.9\textwidth]{trentinoRegNatTrend}
	\caption{Trentino Alto Adige and Italy trends}
	\label{trentinoRegNatTrend}
\end{figure}

\begin{figure}[hp]
	\centering
	\includegraphics[width=0.9\textwidth]{trentinoClassiPot}
	\caption{Trentino PV installed power, grouped by plant class}
	\label{trentinoClassiPot}
\end{figure}

\subsection*{Piemonte}

\begin{figure}[hp]
	\centering
	\includegraphics[width=0.9\textwidth]{piemonteRegTrend}
	\caption{Piemonte PV installed power}
	\label{piemonteRegTrend}
\end{figure}

\begin{figure}[hp]
	\centering
	\includegraphics[width=0.9\textwidth]{piemonteRegNatTrend}
	\caption{Piemonte and Italy trends}
	\label{piemonteRegNatTrend}
\end{figure}

\begin{figure}[hp]
	\centering
	\includegraphics[width=0.9\textwidth]{piemonteClassiPot}
	\caption{Piemonte PV installed power, grouped by plant class}
	\label{piemonteClassiPot}
\end{figure}

\clearpage

\subsection*{Lazio}

\begin{figure}[hp]
	\centering
	\includegraphics[width=0.9\textwidth]{lazioRegTrend}
	\caption{Lazio PV installed power}
	\label{lazioRegTrend}
\end{figure}

\begin{figure}[hp]
	\centering
	\includegraphics[width=0.9\textwidth]{lazioRegNatTrend}
	\caption{Lazio and Italy trends}
	\label{lazioRegNatTrend}
\end{figure}

\begin{figure}[hp]
	\centering
	\includegraphics[width=0.9\textwidth]{lazioClassiPot}
	\caption{Lazio PV installed power, grouped by plant class}
	\label{lazioClassiPot}
\end{figure}

\clearpage

\subsection*{Toscana}

\begin{figure}[hp]
	\centering
	\includegraphics[width=0.9\textwidth]{toscanaRegTrend}
	\caption{Toscana PV installed power}
	\label{toscanaRegTrend}
\end{figure}

\begin{figure}[hp]
	\centering
	\includegraphics[width=0.9\textwidth]{toscanaRegNatTrend}
	\caption{Toscana and Italy trends}
	\label{toscanaRegNatTrend}
\end{figure}

\begin{figure}[hp]
	\centering
	\includegraphics[width=0.9\textwidth]{toscanaClassiPot}
	\caption{Toscana PV installed power, grouped by plant class}
	\label{toscanaClassiPot}
\end{figure}

\subsection*{Sicilia}

\begin{figure}[hp]
	\centering
	\includegraphics[width=0.9\textwidth]{siciliaRegTrend}
	\caption{Sicilia PV installed power}
	\label{siciliaRegTrend}
\end{figure}

\begin{figure}[hp]
	\centering
	\includegraphics[width=0.9\textwidth]{siciliaRegNatTrend}
	\caption{Sicilia and Italy trends}
	\label{siciliaRegNatTrend}
\end{figure}

\begin{figure}[hp]
	\centering
	\includegraphics[width=0.9\textwidth]{siciliaClassiPot}
	\caption{Sicilia PV installed power, grouped by plant class}
	\label{siciliaClassiPot}
\end{figure}

\subsection*{Friuli Venezia Giulia}

\begin{figure}[hp]
	\centering
	\includegraphics[width=0.9\textwidth]{friuliRegTrend}
	\caption{Friuli Venezia Giulia PV installed power}
	\label{friuliRegTrend}
\end{figure}

\begin{figure}[hp]
	\centering
	\includegraphics[width=0.9\textwidth]{friuliRegNatTrend}
	\caption{Friuli Venezia Giulia and Italy trends}
	\label{friuliRegNatTrend}
\end{figure}

\begin{figure}[hp]
	\centering
	\includegraphics[width=0.9\textwidth]{friuliClassiPot}
	\caption{Friuli Venezia Giulia PV installed power, grouped by plant class}
	\label{friuliClassiPot}
\end{figure}

\clearpage

\subsection*{Puglia}

\begin{figure}[hp]
	\centering
	\includegraphics[width=0.9\textwidth]{pugliaRegTrend}
	\caption{Puglia PV installed power}
	\label{pugliaRegTrend}
\end{figure}

\begin{figure}[hp]
	\centering
	\includegraphics[width=0.9\textwidth]{pugliaRegNatTrend}
	\caption{Puglia and Italy trends}
	\label{pugliaRegNatTrend}
\end{figure}

\begin{figure}[hp]
	\centering
	\includegraphics[width=0.9\textwidth]{pugliaClassiPot}
	\caption{Puglia PV installed power, grouped by plant class}
	\label{pugliaClassiPot}
\end{figure}

\subsection*{Marche}

\begin{figure}[hp]
	\centering
	\includegraphics[width=0.9\textwidth]{marcheRegTrend}
	\caption{Marche PV installed power}
	\label{marcheRegTrend}
\end{figure}

\begin{figure}[hp]
	\centering
	\includegraphics[width=0.9\textwidth]{marcheRegNatTrend}
	\caption{Marche and Italy trends}
	\label{marcheRegNatTrend}
\end{figure}

\begin{figure}[hp]
	\centering
	\includegraphics[width=0.9\textwidth]{marcheClassiPot}
	\caption{Marche PV installed power, grouped by plant class}
	\label{marcheClassiPot}
\end{figure}

\clearpage

\subsection*{Campania}

\begin{figure}[hp]
	\centering
	\includegraphics[width=0.9\textwidth]{campaniaRegTrend}
	\caption{Campania PV installed power}
	\label{campaniaRegTrend}
\end{figure}

\begin{figure}[hp]
	\centering
	\includegraphics[width=0.9\textwidth]{campaniaRegNatTrend}
	\caption{Campania and Italy trends}
	\label{campaniaRegNatTrend}
\end{figure}

\begin{figure}[hp]
	\centering
	\includegraphics[width=0.9\textwidth]{campaniaClassiPot}
	\caption{Campania PV installed power, grouped by plant class}
	\label{campaniaClassiPot}
\end{figure}

\subsection*{Basilicata}

\begin{figure}[hp]
	\centering
	\includegraphics[width=0.9\textwidth]{basilicataRegTrend}
	\caption{Basilicata PV installed power}
	\label{basilicataRegTrend}
\end{figure}

\begin{figure}[hp]
	\centering
	\includegraphics[width=0.9\textwidth]{basilicataRegNatTrend}
	\caption{Basilicata and Italy trends}
	\label{basilicataRegNatTrend}
\end{figure}

\begin{figure}[hp]
	\centering
	\includegraphics[width=0.9\textwidth]{basilicataClassiPot}
	\caption{Basilicata PV installed power, grouped by plant class}
	\label{basilicataClassiPot}
\end{figure}

\subsection*{Sardegna}

\begin{figure}[hp]
	\centering
	\includegraphics[width=0.9\textwidth]{sardegnaRegTrend}
	\caption{Sardegna PV installed power}
	\label{sardegnaRegTrend}
\end{figure}

\begin{figure}[hp]
	\centering
	\includegraphics[width=0.9\textwidth]{sardegnaRegNatTrend}
	\caption{Sardegna and Italy trends}
	\label{sardegnaRegNatTrend}
\end{figure}

\begin{figure}[hp]
	\centering
	\includegraphics[width=0.9\textwidth]{sardegnaClassiPot}
	\caption{Sardegna PV installed power, grouped by plant class}
	\label{sardegnaClassiPot}
\end{figure}

\clearpage

\subsection*{Liguria}

\begin{figure}[hp]
	\centering
	\includegraphics[width=0.9\textwidth]{liguriaRegTrend}
	\caption{Liguria PV installed power}
	\label{liguriaRegTrend}
\end{figure}

\begin{figure}[hp]
	\centering
	\includegraphics[width=0.9\textwidth]{liguriaRegNatTrend}
	\caption{Liguria and Italy trends}
	\label{liguriaRegNatTrend}
\end{figure}

\begin{figure}[hp]
	\centering
	\includegraphics[width=0.9\textwidth]{liguriaClassiPot}
	\caption{Liguria PV installed power, grouped by plant class}
	\label{liguriaClassiPot}
\end{figure}

\subsection*{Calabria}

\begin{figure}[hp]
	\centering
	\includegraphics[width=0.9\textwidth]{calabriaRegTrend}
	\caption{Calabria PV installed power}
	\label{calabriaRegTrend}
\end{figure}

\begin{figure}[hp]
	\centering
	\includegraphics[width=0.9\textwidth]{calabriaRegNatTrend}
	\caption{Calabria and Italy trends}
	\label{calabriaRegNatTrend}
\end{figure}

\begin{figure}[hp]
	\centering
	\includegraphics[width=0.9\textwidth]{calabriaClassiPot}
	\caption{Calabria PV installed power, grouped by plant class}
	\label{calabriaClassiPot}
\end{figure}

\subsection*{Abruzzo}

\begin{figure}[hp]
	\centering
	\includegraphics[width=0.9\textwidth]{abruzzoRegTrend}
	\caption{Abruzzo PV installed power}
	\label{abruzzoRegTrend}
\end{figure}

\begin{figure}[hp]
	\centering
	\includegraphics[width=0.9\textwidth]{abruzzoRegNatTrend}
	\caption{Abruzzo and Italy trends}
	\label{abruzzoRegNatTrend}
\end{figure}

\begin{figure}[hp]
	\centering
	\includegraphics[width=0.9\textwidth]{abruzzoClassiPot}
	\caption{Abruzzo PV installed power, grouped by plant class}
	\label{abruzzoClassiPot}
\end{figure}

\subsection*{Molise}

\begin{figure}[hp]
	\centering
	\includegraphics[width=0.9\textwidth]{moliseRegTrend}
	\caption{Molise PV installed power}
	\label{moliseRegTrend}
\end{figure}

\begin{figure}[hp]
	\centering
	\includegraphics[width=0.9\textwidth]{moliseRegNatTrend}
	\caption{Molise and Italy trends}
	\label{moliseRegNatTrend}
\end{figure}

\begin{figure}[hp]
	\centering
	\includegraphics[width=0.9\textwidth]{moliseClassiPot}
	\caption{Molise PV installed power, grouped by plant class}
	\label{moliseClassiPot}
\end{figure}

\subsection*{Valle d'Aosta}

\begin{figure}[hp]
	\centering
	\includegraphics[width=0.9\textwidth]{vdaostaRegTrend}
	\caption{Valle d'Aosta PV installed power}
	\label{vdaostaRegTrend}
\end{figure}

\begin{figure}[hp]
	\centering
	\includegraphics[width=0.9\textwidth]{vdaostaRegNatTrend}
	\caption{Valle d'Aosta and Italy trends}
	\label{vdaostaRegNatTrend}
\end{figure}

\begin{figure}[hp]
	\centering
	\includegraphics[width=0.9\textwidth]{vdaostaClassiPot}
	\caption{Valle d'Aosta PV installed power, grouped by plant class}
	\label{vdaostaClassiPot}
\end{figure}

\end{document}
